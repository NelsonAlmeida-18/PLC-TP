\documentclass[11pt,a4paper]{report}%especifica o tipo de documento que tenciona escrever: carta, artigo, relatório... neste caso é um relatório
% [11pt,a4paper] Define o tamanho principal das letras do documento. caso não especifique uma delas, é assumido 10pt
% a4paper -- Define o tamanho do papel.

\usepackage[portuges]{babel}%Babel -- irá activar automaticamente as regras apropriadas de hifenização para a língua todo o
                                   %-- o texto gerado é automaticamente traduzido para Português.
                                   %  Por exemplo, “chapter” irá passar a “capítulo”, “table of contents” a “conteúdo”.
                                   % portuges -- específica para o Português.
\usepackage[utf8]{inputenc} % define o encoding usado texto fonte (input)--usual "utf8" ou "latin1

\usepackage{graphicx} %permite incluir graficos, tabelas, figuras
\usepackage{url} % para utilizar o comando \url{}
\usepackage{enumerate} %permite escolher, nas listas enumeradas, se os iems sao marcados com letras ou s-romanos em vez de numeracao normal

%\usepackage{apalike} % gerar biliografia no estilo 'named' (apalike)

\usepackage{color} % Para escrever em cores

\usepackage{multirow} %tabelas com multilinhas
\usepackage{array} %formatação especial de tabelas em array

\usepackage[pdftex]{hyperref} % transformar as referências internas do seu documento em hiper-ligações.

%Exemplos de fontes -- nao e vulgar mudar o tipo de fonte
%\usepackage{tgbonum} % Fonte de letra: TEX Gyre Bonum
%\usepackage{lmodern} % Fonte de letra: Latin Modern Sans Serif
%\usepackage{helvet}  % Fonte de letra: Helvetica
%\usepackage{charter} % Fonte de letra:Charter

\definecolor{saddlebrown}{rgb}{0.55, 0.27, 0.07} % para definir uma nova cor, neste caso 'saddlebrown'

\usepackage{listings}  

\definecolor{codegreen}{rgb}{0,0.6,0}
\definecolor{codegray}{rgb}{0.5,0.5,0.5}
\definecolor{codeblue}{rgb}{0,0,255}
\definecolor{backcolour}{rgb}{0.95,0.95,0.92}
\definecolor{codefirebrick}{rgb}{160,0,0}

\lstdefinestyle{mystyle}{
	backgroundcolor=\color{backcolour},   
	commentstyle=\color{codegreen},
	keywordstyle=\color{codefirebrick},
	numberstyle=\tiny\color{codegray},
	stringstyle=\color{codeblue},
	basicstyle=\ttfamily\footnotesize,
	breakatwhitespace=false,         
	breaklines=true,                 
	captionpos=b,                    
	keepspaces=true,                 
	numbers=left,                    
	numbersep=5pt,                  
	showspaces=false,                
	showstringspaces=false,
	showtabs=false,                  
	tabsize=2
}

\lstset{style=mystyle}

\usepackage{xspace} % deteta se a seguir a palavra tem uma palavra ou um sinal de pontuaçao se tiver uma palavra da espaço, se for um sinal de pontuaçao nao da espaço

\usepackage{hyperref}

\parindent=0pt %espaço a deixar para fazer a  indentação da primeira linha após um parágrafo
\parskip=2pt % espaço entre o parágrafo e o texto anterior

\setlength{\oddsidemargin}{-1cm} %espaço entre o texto e a margem
\setlength{\textwidth}{18cm} %Comprimento do texto na pagina
\setlength{\headsep}{-1cm} %espaço entre o texto e o cabeçalho
\setlength{\textheight}{23cm} %altura do texto na pagina

% comando '\def' usado para definir abreviatura (macros)
% o primeiro argumento é o nome do novo comando e o segundo entre chavetas é o texto original, ou sequência de controle, para que expande
\def\darius{\textsf{Darius}\xspace}
\def\antlr{\texttt{AnTLR}\xspace}
\def\pe{\emph{Publicação Eletrónica}\xspace}
\def\titulo#1{\section{#1}}    %no corpo do documento usa-se na forma '\titulo{MEU TITULO}'
\def\super#1{{\em Supervisor: #1}\\ }
\def\area#1{{\em \'{A}rea: #1}\\[0.2cm]}
\def\resumo{\underline{Resumo}:\\ }





\title{Processamento de Linguagens e Compiladores (3º Ano)\\
       \textbf{Trabalho Prático 1}\\ Relatório de Desenvolvimento
       }
\author{Bruno Fernandes\\ (a95972)\\\\ \includegraphics[width=4cm]{Imagens/Bruno} \and Nelson Almeida\\ (a95652)\\\\ \includegraphics[width=4cm]{Imagens/Nelson}
         \and Nuno Costa\\ (a97610)\\\\ \includegraphics[width=4cm]{Imagens/Nuno}
       }
\date{01/11/2022}

\begin{document} % corpo do documento
\maketitle % apresentar titulo, autor e data


\tableofcontents


\chapter{Introdução} \label{chap:intro} %referência cruzada
No âmbito da disciplina de Processamentos de Linguagens e Compiladores foi-nos proposto pelo docente Pedro Rangel Henriques um trabalho de pesquisa cujo objetivo principal é consolidar a aprendizagem da utilização de linguagens regulares em programas em linguagem Python.
\\
\\
Decidimos então resolver os exercícios 1 e 2.
\\
O exercício 1 tem como intuito a construção de vários programas em Python para processar o ficheiro de texto "processos.txt" com a intenção de calcular as frequências de alguns elemetos pedidos ao longo do exercício.
\\
O exercício 2 tem como objetivo a construção de vários programas para processar o dataset "emd.csv" e produzir o que é solicitado durante o exercício.
\\
\\
Neste documento apresentamos a nossa resolução para cada um dos problemas.


\newpage


\section*{Estrutura do Relatório}
O relatório está organizado da seguinte forma:\\
Começamos por fazer uma pequena introdução, capítulo 1, onde referimos o objetivo deste trabalho e explicamos resumidamente o intuito de cada problema que decidimos resolver.\\
No capítulo 2 aprensentamos o enunciado dos problemas escolhidos.\\
O capítulo 3 contém a resolução do primeiro problema e no 4 a resolução do segundo.\\
No capítulo 5 mostramos os resultados obtidos.\\
E, por fim, o ultimo capítulo contém a conclusão do trabalho realizado.



\newpage



\chapter{Análise e Especificações}

\section{Processador de Pessoas listadas nos Róis de Confessados}
Construa agora um ou vários programas Python para processar o texto 'processos.txt' com o intuito de calcular frequências de alguns elementos (a ideia é utilizar arrays associativos para o efeito) conforme solicitado a seguir:
\begin{enumerate}[a)]
	\item Calcula a frequência de processos por ano (primeiro elemento da data);
	\item Calcula a frequência de nomes próprios (o primeiro em cada nome) e apelidos (o ultimo em cada nome) por séculos;
	\item Calcula a frequência dos vários tipos de relação: irmão, sobrinho, etc.
	\item Imprimir os 20 primeiros registos num novo cheiro de output mas em formato Json.
\end{enumerate}


\section{Processador de Registos de Exames Médicos Desportivos}
Neste exercício pretende-se trabalhar com um dataset gerado no âmbito do registo de exames médicos desportivos.
Construa, então, um ou vários programas Python para processar o dataset "emd.csv" e produzir o solicitado nas alíneas seguintes:
\begin{enumerate}[1.]
	\item Página principal: de nome "index.html", contendo os seguintes indicadores estatísticos:
	\begin{enumerate}[a)]
		\item Datas extremas dos registos no dataset;
		\item Distribuição por modalidade em cada ano e no total;
		\item Distribuição por idade e género (para a idade, considera apenas 2 escalões: < 35 anos e >= 35);
		\item Distribuição por morada;
		\item Percentagem de aptos e não aptos por ano.
	\end{enumerate}
	\item Página do indicador: clicando no indicador na página principal, devemos saltar para a página do indicador
	onde temos a informação que permitiu obter esse indicador. Por exemplo, para a distribuição por morada,
	a página deverá apresentar uma lista de moradas, ordenada alfabeticamente e para cada morada deverá
	apresentar uma sublista de registos, ordenada alfabeticamente por nome de atleta (com os dados: nome do
	atleta, modalidade).
\end{enumerate}



\chapter{Resolução do Exercício 1}

Inicialmente criamos um ficheiro principal "main.py", onde instalamos os packages necessários para a execução do programa pedindo ao utilizador o caminho (path) para o ficheiro que pretende utilizar, tendo como pré-definido o ficheiro "processos.txt".
Neste mesmo ficheiro criamos a função "parser", que recebe como parâmetro o ficheiro pretendido, gerando um conjuntos de arrays com o conteúdo do mesmo que será usado posteriormente para cada uma das 4 alíneas.

\begin{enumerate}[a)]
	\item Para resolver a primeira alínea, desenvolvemos o seguinte código para calcular a frequência de processos por ano.
	\begin{lstlisting}[language=python]
		def __init__(self, processos):
			print(self.processesPerTimeSpan(processos))
		
		def processesPerTimeSpan(self, processos):
			processosPorData = {}
			while (processos != []):
				ano = re.match("(([0-9]{1,4})-*)", processos[0][1]).group(2)
			processosNoAno, processosNoutrosAnos = self.filtraPorAno(processos, ano)
			processosPorData[ano] = processosNoAno
			processos = processosNoutrosAnos
			return processosPorData
		
		def filtraPorAno(self, processos, yearToFilter):
			procPerYear = 0
			processosNoutrosAnos = []
			for pessoa in processos:
				flag = 0
				for elem in pessoa:
					if re.search(yearToFilter+r"-[0-9]{2}-[0-9]{1,2}", elem):
						flag = 1
				if flag == 1:
					procPerYear += 1
				else:
					processosNoutrosAnos.append(pessoa)
			return (procPerYear, processosNoutrosAnos)
	\end{lstlisting}
	
	Começamos por criar uma função "processesPerTimeSpan" para obter um dicionário com  a frequência de processos para cada ano.\\
	Para descobrir cada ano fizemos uso da função match com a expressão regular \textbf{"(([0-9]\{1,4\})-*)"} e invocamos a função auxiliar "filtraPorAno".
	\begin{lstlisting}[language=python]
		while (processos != []):
			ano = re.match("(([0-9]{1,4})-*)", processos[0][1]).group(2)
			processosNoAno, processosNoutrosAnos = self.filtraPorAno(processos, ano)
	\end{lstlisting}
	
	A função auxiliar "filtraPorAno" tem como resultado o número de processos do ano passado como argumento e uma lista contendo todos os processos restantes existentes no ficheiro.\\
	Assim sendo, para cada processo utilizamos a função search com a expressão regular \textbf{"\--[0-9]\{2\}-[0-9]\{1,2\}"} para encontrar todas as datas desse ano passado com argumento.
	\begin{lstlisting}[language=python]
		for pessoa in processos:
			flag = 0
			for elem in pessoa:
				if re.search(yearToFilter+r"-[0-9]{2}-[0-9]{1,2}", elem):
					flag = 1
			if flag == 1:
				procPerYear += 1
			else:
				processosNoutrosAnos.append(pessoa)
	\end{lstlisting}
	
	
	\newpage
	
	
	\item Para a resolução desta alínea, desenvolvemos o seguinte código de modo a obter a frequência de nomes próprios e apelidos por século.
	\begin{lstlisting}[language=python]
		def __init__(self, processos):
			print(self.nomesPorSeculo(processos))
		
		def nomesPorSeculo(self, processos):
			dic = {"firstNames": {}, "lastNames": {}}
			for processo in processos:
				ano = re.match("(([0-9]{1,4})-*)", processo[1]).group(2)
				sec = int(ano)//100 + 1
				if sec not in dic["firstNames"]:
					dic["firstNames"][sec] = {}
					dic["lastNames"][sec] = {}
				i = 2
				while (i <= 4):
					names = re.split(" ", processo[i])
					fstName = names[0]
					lastName = names[-1]
					if fstName not in dic["firstNames"][sec]:
						dic["firstNames"][sec][fstName] = 0
					if lastName not in dic["lastNames"][sec]:
						dic["lastNames"][sec][lastName] = 0
					dic["firstNames"][sec][fstName] += 1
					dic["lastNames"][sec][lastName] += 1
					i += 1
			return dic
	\end{lstlisting}

	Começamos por criar uma função "nomesPorSeculo" com o objetivo de obter um dicionário com duas chaves, uma para os nomes próprios e a outra para os apelidos.\\
	Para tal começamos por encontrar o ano em cada processo usando a função match e a expressão regular \textbf{"(([0-9]\{1,4\})-*)"} e deste modo sabemos a que século pertence esse ano adicionando-o,caso nao exista, ao dicionário.
	\begin{lstlisting}[language=python]
			for processo in processos:
				ano = re.match("(([0-9]{1,4})-*)", processo[1]).group(2)
				sec = int(ano)//100 + 1
				if sec not in dic["firstNames"]:
					dic["firstNames"][sec] = {}
					dic["lastNames"][sec] = {}
	\end{lstlisting}
	Para obter o nome próprio e o apelido recorremos à separação dos nomes de cada pessoa por utilizando a função split, partindo-os nomes pelos espaços existentes entre cada um.\\
	Após isso adicionamos os nomes próprios e apelidos aos respetivos séculos.
	\begin{lstlisting}[language=python]
		names = re.split(" ", processo[i])
		fstName = names[0]
		lastName = names[-1]
		if fstName not in dic["firstNames"][sec]:
			dic["firstNames"][sec][fstName] = 0
		if lastName not in dic["lastNames"][sec]:
			dic["lastNames"][sec][lastName] = 0
		dic["firstNames"][sec][fstName] += 1
		dic["lastNames"][sec][lastName] += 1
	\end{lstlisting}
	
	
	\newpage
	
	
	\item Para a resolução deste problema, desenvolvemos o código que se segue e calcula a frequência dos vários tipos de relação.
	
	\begin{lstlisting}[language=python]
		def __init__(self, processos):
			print(self.tiposParentesco(processos))
		
		def tiposParentesco(self, pessoas):
			tipos = {}
			for pessoa in pessoas:
				for elem in pessoa:
					parent = re.findall(r'(?<=[a-z]\,)([PMTISANB][A-Za-z ]+|Filho|Filhos)(?=\.)', elem)
					for x in parent:
						if x in tipos.keys():
							tipos[x] = tipos.get(x, 0) + 1
						else:
							tipos[x] = 1
	\end{lstlisting}
	Começamos por criar uma função "tiposParentesco" que tem como retorno um dicionário que a cada tipo de parentesco associa o número de vezes que o mesmo aparece no ficheiro "processos.txt".\\
	Primeiramente começamos por identificar cada grau de parentesco encontrado no ficheiro. E visto que podem existir mais do que um por cada pessoa, recorremos à função findall com a expressão regular \textbf{"(?$<=$[a-z]$\backslash$,)([PMTISANB][A-Za-z ]+$|$Filho$|$Filhos)(?=$\backslash$.)}".
	\begin{lstlisting}[language=python]
		tipos = {}
		for pessoa in pessoas:
			for elem in pessoa:
				parent = re.findall(r'(?<=[a-z]\,)([PMTISANB][A-Za-z ]+|Filho|Filhos)(?=\.)', elem)   
	\end{lstlisting}
	
	
	\newpage
	
	
	\item Para a resolução da última alínea, pretendiamos imprimir os 20 primeiros registos num novo ficheiro de output mas em formato Json e para tal desenvolvemos o código abaixo.\\
	
	\begin{lstlisting}[language=python]
		def __init__(self, processos, limit):
			if (self.jsonFormatter(processos[0:limit])):
				print("Ficheiro criado com sucesso")
			else:
				print("Ocorreu um erro a criar o ficheiro")
		
		def jsonFormatter(self, pessoas):
			file = open("./TP1/ex1/rois.json", "w")
			file.write("{\"data\":[")
			templateJson = '''{"folder_id":#,"date":#,"name":#,"father":#,"mother":#,"obs":#}'''
			pos = 0
			for pessoa in pessoas:
				pessoa.append("")
				person = templateJson
				for i in pessoa:
					person = re.sub("#", f"\"{i}\"", person, count=1)
				if (pos < len(pessoas)-1):
					file.write(person+",\n")
				else:
					file.write(person)
				pos += 1
			file.write("]}")
			file.close()
			return True   
	\end{lstlisting}
	Inicialmente criamos a função "jsonFormatter" que irá retornar o novo output em formato Json.\\
	Esta função começa por escrever de acordo com o template Json no ficheiro criado por nós "rois.json" e retornar em caso de sucesso "Ficheiro criado com sucesso".

\end{enumerate}


\newpage


\chapter{Resolução do Exercício 2}

Começamos por criar um ficheiro principal "main.py" onde se começa instala os packages necessários para a execução do programa e pede ao utilizador o path para o ficheiro que pretende utilizar, tendo como pré-definido o ficheiro "emd.csv".
Com a função "parser" recebendo como parâmetro o ficheiro pretendido, obtemos um conjuntos de arrays com o conteúdo do mesmo que é usado posteriormente para cada uma das 5 alíneas.

\begin{enumerate}
	\item Para exercício 1 criamos um ficheiro "index.html" que faz referência às restantes alíneas.\\
	\begin{lstlisting}[language=html]
		<!DOCTYPE html>
		<html lang="en">
		<head>
			<meta charset="UTF-8">
			<meta http-equiv="X-UA-Compatible" content="IE=edge">
			<meta name="viewport" content="width=device-width, initial-scale=1.0">
			<title>Ex2 TP1 PLC</title>
			<link rel="stylesheet" href="main.css">
		</head>
		<body>
			<div className="mainBar" style="padding-top: 50vh;">
				<a href="a.html">A</a>
				<a href="b.html">B</a>
				<a href="c.html">C</a>
				<a href="d.html">D</a>
				<a href="e.html">E</a>
			</div>
		</body>
		</html>
	\end{lstlisting}
	Para as alíneas b), c), d), e) e f) utilizamos a biblioteca "pandas" para fazer a função "htmlGenerator" para através de um dataframe gerar a representação da tabela em html, esta é guardada numa variável para depois ser processada.\\
	Utilizamos também a biblioteca "matplotlib" para podermos gerar gráficos.\\
	Para adicionar o título da pagina, os gráficos gerados e o "inline styling" ao codigo HTML gerado, utilizamos o re.sub que faz a substituição da tag table gerada por o que pretendemos adicionar.
		
	\newpage
	
	\begin{enumerate}[a)]
		\item Para resolver a primeira alínea, criamos um array com todas as datas do ficheiro selecionado e determinamos o mínimo e o máximo deste.\\
		\begin{lstlisting}[language=python]
		def a(self, file):
			datas = [i[2] for i in file]
			file = open("./TP1/ex2/1/website/a.html", "w")
			templateText = 
				'''<!DOCTYPE html >
				<html lang = "en" >
				<head >
					<meta charset = "UTF-8" >
					<meta http-equiv = "X-UA-Compatible" 	content = "IE=edge" >
					<meta name = "viewport" content = "width=device-width, initial-scale=1.0" >
					<title >a</title >
					<link rel="stylesheet" href="./main.css">
				</head >
				<body >
				</body >
				</html >'''
			newText = re.sub(
				"<body >", f'''\n<body >\n<p class="title">Datas extremas dos registos no dataset</p>\n<p class="minData">Extremo inferior do dataset: {min(datas)}</p>\n<p class="maxData">Extremo superior do dataset: {max(datas)}</p>''', templateText)
			file.write(newText)
			file.close()
		\end{lstlisting}
	
		\newpage
		
		\item Para a resolução da alínea b) e assim determinar a distribuição por modalidade em cada ano e no total, desenvolvemos o seguinte código.\\
		\begin{lstlisting}[language=python]	
				def b(self, data):
					sportsPerYear, totalSports = self.sportsPerYearandTotal(data)
					indexes = sorted(sportsPerYear)
					sportsPerYear = {i: sportsPerYear[i] for i in indexes}
					totalSports = dict(sorted(totalSports.items(), key=lambda x: x[1]))
					self.plotter(sportsPerYear, totalSports)
					self.htmlGenerator(sportsPerYear, totalSports)
				
				def sportsPerYearandTotal(self, data):
					sportsPerYear = {}
					totalSports = {}
					while (data):
						year = re.match("(([0-9]{1,4})-*)", data[0][2]).group(2)
						sportsYear, updatedData = self.sportsPerYear(data, year)
						sportsPerYear[year] = sportsYear
						data = updatedData
				
					for year in sportsPerYear:
						for sport in sportsPerYear[year]:
							if sport not in totalSports:
								totalSports[sport] = 0
							totalSports[sport] += sportsPerYear[year][sport]
					return sportsPerYear, totalSports
				
				def sportsPerYear(self, data, yearToFilter):
					sportsPerY = {}
					newData = []
					for person in data:
						if re.match(yearToFilter+r"(-[0-9]{1,2}){2}", person[2]):
							sport = person[8]
							if sport not in sportsPerY:
								sportsPerY[sport] = 0
							sportsPerY[sport] += 1
						else:
							newData.append(person)
					return sportsPerY, newData
				
				def htmlGenerator(self, data1, data2):
					file = open(f"./TP1/ex2/1/website/b.html", "w")
					templateText1 = pd.DataFrame(data1).to_html()
					templateText1 = re.sub( 
						r'''<table border="1" class="dataframe">''', '''<h1 class="title">Distribuição por modalidade em cada ano e no total</h1>\n<div class="images" style="display: flex; width: 100%; padding-bottom: 2rem;"><img src="./src/b1.png" alt="" style="width:50%">\n<img src="./src/b2.png" alt="" style="width:50%">\n</div>\n<link rel="stylesheet" href="./main.css">\n<table border="1" class="dataframe">''', templateText1
					)
					file.write(templateText1)
					templateText2 = pd.DataFrame(data2, index=["Praticantes"]).to_html()
					file.write(templateText2)
					file.close()
				
				def plotter(self, data1, data2):
					df1 = pd.DataFrame(data1)
					df1.plot(kind="bar")
					plt.savefig("./TP1/ex2/1/website/src/b1.png")
					df2 = pd.DataFrame(data2, index=["Praticantes"])
					df2.plot(kind="bar")
					plt.savefig("./TP1/ex2/1/website/src/b2.png")
		\end{lstlisting}
		
		Começamos por criar a função auxiliar "sportsPerYearandTotal" para obtermos dois dicionários, um com os desportos distribuídos por anos e outro com todos os desportos existentes no ficheiro.\\
		Para cada pessoa existente no ficheiro utilizamos o match com a expressão regular \textbf{"(([0-9]{1,4})-*)"} para encontrar o ano que vai ser utilizado como parâmetro em outra função auxiliar "sportsPerYear".
		\begin{lstlisting}[language=python]	
			year = re.match("(([0-9]{1,4})-*)", data[0][2]).group(2)
			sportsYear, updatedData = self.sportsPerYear(data, year)
		\end{lstlisting}
		Na função "sportsPerYear" começamos por criar um dicionário com os desportos existentes nesse ano e a sua frequência.
		\begin{lstlisting}[language=python]	
			if re.match(yearToFilter+r"(-[0-9]{1,2}){2}", person[2]):
				sport = person[8]
				if sport not in sportsPerY:
					sportsPerY[sport] = 0
				sportsPerY[sport] += 1
		\end{lstlisting}
		E um array onde vão ser adicionados apenas os dados relativos aos restantes anos (diferentes do ano passado como parâmetro), para assim, evitar repetições e garantir um melhor controlo.
		\begin{lstlisting}[language=python]	
			else:
				newData.append(person)
		\end{lstlisting}
		No fim vão ser retornados o dicionário "sportsPerY" e o array "newData" que são utilizados para atualizar o dicionário "sportsPerYear" e os dados do array "data" do método anterior.
		\begin{lstlisting}[language=python]	
			sportsPerYear[year] = sportsYear
			data = updatedData
		\end{lstlisting}
		Por último, com os dados colecionados no dicionário "sportsPerYear", soma o número de desportistas para cada ano e para cada desporto e coleciona no dicionário de "totalSports".
		\begin{lstlisting}[language=python]	
			for year in sportsPerYear:
				for sport in sportsPerYear[year]:
					if sport not in totalSports:
						totalSports[sport] = 0
					totalSports[sport] += sportsPerYear[year][sport]
		\end{lstlisting}
		
		\newpage
		
		\item Para a resolução da alínea c) e assim determinar a distribuição por idade e género, desenvolvemos o seguinte código.
		\begin{lstlisting}[language=python]	
			def c(self, processos):
				data = processos[1:]  # aqui ficam tudo menos os indicadores
				data = self.distByAgeAndGender(data)
				self.plotter(data)
				self.htmlGenerator(data)
			
			def distByAgeAndGender(self, data):
				dist = {"<35": {}, ">=35": {}}
				ageIndex = 5
				genderIndex = 6
				for person in data:
					ageRange = ">=35"
					if int(person[ageIndex]) < 35:
						ageRange = "<35"
					if person[genderIndex] not in dist[ageRange]:
						dist[ageRange][person[genderIndex]] = 0
					dist[ageRange][person[genderIndex]] += 1
			return dist
			
			def htmlGenerator(self, data1):
				file = open(f"./TP1/ex2/1/website/c.html", "w")
				templateText = pd.DataFrame(data1).to_html()
				templateText = re.sub(
					r'''<table border="1" class="dataframe">''', '''<p class="title">Distribuição por idade e género</p>\n<div class="images" style="padding-bottom: 2rem;"><img src="./src/c.png" alt="">\n</div>\n<link rel="stylesheet" href="./main.css">\n<table border="1" class="dataframe">''', templateText)
				file.write(templateText)
				file.close()
			
			def plotter(self, data):
				df1 = pd.DataFrame(data)
				df1.plot(kind="bar")
				plt.savefig("./TP1/ex2/1/website/src/c.png")
		\end{lstlisting}
	
		Criamos a função auxiliar "distByAgeAndGender" que começa por criar um dicionário onde vão ser guardadas as informações, vai percorrer cada pessoa do ficheiro, verificar se a sua idade é maior ou menor que 35 anos, verificar o género dessa pessoa e incrementar na contação.
		
		\newpage
		
		\item Para a resolução da alínea d) e assim determinar a distribuição por morada, desenvolvemos o seguinte código.
		\begin{lstlisting}[language=python]
			def d(self, processos):
				data = processos[1:]
				data = self.distByAddress(data)
				self.plotter(data)
				self.htmlGenerator(data)
			
			def distByAddress(self, data):
				distByAddress = {}
				for person in data:
					address = person[7]
					if address not in distByAddress:
						distByAddress[address] = 0
						distByAddress[address] += 1
				return distByAddress
			
			def htmlGenerator(self, data1):
				file = open(f"./TP1/ex2/1/website/d.html", "w")
				templateText = pd.DataFrame(data1, index=[""]).to_html()
				templateText = re.sub(
					r'''<table border="1" class="dataframe">''', '''<p class="title">Distribuição por morada</p>\n<div class="images" style="padding-bottom: 2rem;"><img src="./src/d.png" alt="">\n</div>\n<link rel="stylesheet" href="./main.css">\n<table border="1" class="dataframe">''', templateText
				)
				file.write(templateText)
				file.close()
			
			def plotter(self, data):
				if len(data) >= 50:
					df1 = pd.DataFrame(data, index=[""])
					df1.plot(kind="bar")
					plt.savefig("./TP1/ex2/1/website/src/d.png")
				else:
					df1 = pd.DataFrame(data, index=[""])
					df1.plot(kind="bar")
					plt.savefig("./TP1/ex2/1/website/src/d.png")
		\end{lstlisting}
	
		Criamos a função auxiliar "distByAddress" que começa por criar um dicionário onde vão ser guardadas as informaçõe, vai percorrer cada pessoa do ficheiro, adicionar ao dicionário a sua morada caso esta não exista e incrementar o número de pessoas que vivem nessa morada.
		
		\newpage
		
		\item Para a resolução da alínea e) e assim determinar a percentagem de pessoas aptas e não aptas por ano, desenvolvemos o seguinte código.
		\begin{lstlisting}[language=python]
			def e(self, data):
				data = data[1:]
				readinessStats = {}
				while (data):
					year = re.match("(([0-9]{1,4})-*)", data[0][2]).group(2)
					(ready, total), newData = self.readinessPerYear(data, year)
					readinessPerc = (ready/total)
					readinessStats[year] = {
						"ready": float("{:.2f}".format(100*readinessPerc)),
						"not_ready": float("{:.2f}".format(100*(1-readinessPerc)))
					}
					data = newData
				self.plotter(readinessStats)
				self.htmlGenerator(readinessStats)
			
			def readinessPerYear(self, data, year):
				ready = 0
				total = 0
				newData = []
				for person in data:
					if re.match(year+r"(-[0-9]{1,2}){2}", person[2]):
						if person[-1] == "true":
							ready += 1
						total += 1
					else:
						newData.append(person)
				return (ready, total), newData
			
			def htmlGenerator(self, data1):
				file = open(f"./TP1/ex2/1/website/e.html", "w")
				templateText = pd.DataFrame(data1).to_html()
				templateText = re.sub(
					r'''<table border="1" class="dataframe">''', '''<p class="title">Percentagem de aptos e não aptos por ano</p>\n<div class="images" style="padding-bottom: 2rem;"><img src="./src/e.png" alt="">\n</div>\n<link rel="stylesheet" href="main.css">\n<table border="1" class="dataframe">''', templateText
				)
				file.write(templateText)
				file.close()
			
			def plotter(self, data):
				df1 = pd.DataFrame(data)
				df1.plot(kind="bar")
				plt.savefig("./TP1/ex2/1/website/src/e.png")
		\end{lstlisting}
	
		Começamos por percorrer cada pessoa do ficheiro e utilizando o match com a expressão regular \textbf{"(([0-9]{1,4})-*)"} encontrar o ano que vai ser utilizado como parâmetro na função auxiliar "readinessPerYear".
		\begin{lstlisting}[language=python]
			year = re.match("(([0-9]{1,4})-*)", data[0][2]).group(2)
			(ready, total), newData = self.readinessPerYear(data, year)
		\end{lstlisting}
		Esta função "readinessPerYear" começa por criar duas variáveis "ready" e "total" que vão servir para saber o número de pessoas aptas e o total de pessoas, respetivamente, e um array onde vão ser adicionados apenas os dados relativos aos restantes anos (diferentes do ano passado como parâmetro), para assim, evitar repetições e garantir um melhor controlo.\\
		Começamos por percorrer cada pessoa, verificar se o ano coincide com ano passado como parâmetro, incrementar na variavél "ready" se tal se verificar e essa pessoa estiver apta.
		\begin{lstlisting}[language=python]
			if re.match(year+r"(-[0-9]{1,2}){2}", person[2]):
				if person[-1] == "true":
					ready += 1
				total += 1
		\end{lstlisting}
	
		Na função principal, depois de receber o número de pessoas aptas e no total nesse ano, são calculadas as porcentagens de atletas aptos e não aptos e colocadas no dicionário. Posteriormente os dados são atualizados.
		\begin{lstlisting}[language=python]
			readinessPerc = (ready/total)
			readinessStats[year] = {
				"ready": float("{:.2f}".format(100*readinessPerc)),
				"not_ready": float("{:.2f}".format(100*(1-readinessPerc)))
			}
			data = newData
		\end{lstlisting}
		
	\end{enumerate}
	
\end{enumerate}



\chapter{Testes e Resultados obtidos}

\section{Problema 1}

\subsection{Frequência de Processos Por Ano}

\includegraphics[width=18cm]{Imagens/"1aPrint"}

\subsection{Frequência de nomes próprios e apelidos por séculos}

\includegraphics[width=18cm]{Imagens/"1bPrint"}

\subsection{Frequência dos vários tipos de relação}

\includegraphics[width=18cm]{Imagens/"1cPrint"}

\subsection{Imprimir os 20 primeiros registos em formatos Json}

\begin{lstlisting}
	{
		"data": [
		{
			"folder_id": "575",
			"date": "1894-11-08",
			"name": "Aarao Pereira Silva",
			"father": "Antonio Pereira Silva",
			"mother": "Francisca Campos Silva",
			"obs": ""
		},
		{
			"folder_id": "582",
			"date": "1909-05-12",
			"name": "Abel Almeida",
			"father": "Antonio Manuel Almeida",
			"mother": "Teresa Maria Sousa",
			"obs": ""
		},
		{
			"folder_id": "569",
			"date": "1867-05-23",
			"name": "Abel Alves Barroso",
			"father": "Antonio Alves Barroso",
			"mother": "Maria Jose Alvares Barroso",
			"obs": "Bento Alvares Barroso,Tio Paterno. Proc.32057.   Domingos Jose Alvares Barroso,Tio Materno. Proc.32235."
		},
		{
			"folder_id": "576",
			"date": "1896-11-28",
			"name": "Abel Augusto Oliveira",
			"father": "Francisco Jose Oliveira",
			"mother": "Antonia Rosa Rebelo",
			"obs": "Jose Antonio Oliveira,Irmao. Proc.5020."
		},
		{
			"folder_id": "579",
			"date": "1904-05-27",
			"name": "Abel Gomes Abreu Reis",
			"father": "Antonio Gomes Abreu",
			"mother": "Ana Sequeira Reis",
			"obs": ""
		},
		{
			"folder_id": "579",
			"date": "1904-05-21",
			"name": "Abel Marques Reis",
			"father": "Jose Joaquim Marques Reis",
			"mother": "Bernardina Dantas",
			"obs": ""
		},
		{
			"folder_id": "578",
			"date": "1901-11-12",
			"name": "Abel Martins Pereira",
			"father": "Serafim Martins Pereira",
			"mother": "Emilia Goncalves",
			"obs": ""
		},
		{
			"folder_id": "572",
			"date": "1883-02-01",
			"name": "Abel Pedro Pereira Freitas",
			"father": "Joao Freitas Oliveira",
			"mother": "Cecilia Rosa Pereira",
			"obs": ""
		},
		{
			"folder_id": "578",
			"date": "1900-08-30",
			"name": "Abel Silva Carvalho",
			"father": "Constantino Silva Rego",
			"mother": "Margarida Rosa Carvalho",
			"obs": ""
		},
		{
			"folder_id": "575",
			"date": "1894-04-30",
			"name": "Abelardo Jose Cerqueira Araujo",
			"father": "Jose Maria Araujo",
			"mother": "Leopoldina Cerqueira Ribeiro Araujo",
			"obs": ""
		},
		{
			"folder_id": "575",
			"date": "1894-04-30",
			"name": "Abelardo Jose Cerqueira Araujo",
			"father": "Jose Maria Araujo",
			"mother": "Leopoldina Cerqueira Ribeiro Araujo",
			"obs": ""
		},
		{
			"folder_id": "572",
			"date": "1883-11-24",
			"name": "Abilio Acacio Conceicao Guerreiro",
			"father": "Jose Antonio Pereira Dantas Guerreiro",
			"mother": "Maria Rita Pereira Monteiro",
			"obs": ""
		},
		{
			"folder_id": "579",
			"date": "1902-10-23",
			"name": "Abilio Aires Sousa Pereira Guimaraes",
			"father": "Joaquim Aires Sousa Pereira Guimaraes",
			"mother": "Josefa Rosa Gomes",
			"obs": ""
		},
		{
			"folder_id": "571",
			"date": "1880-01-24",
			"name": "Abilio Antonio Alves",
			"father": "Joao Francisco Alves",
			"mother": "Maria Jesus Rocha",
			"obs": ""
		},
		{
			"folder_id": "573",
			"date": "1889-12-03",
			"name": "Abilio Augusto Arantes",
			"father": "Sebastiao Arantes",
			"mother": "Maria Sousa",
			"obs": ""
		},
		{
			"folder_id": "581",
			"date": "1908-08-11",
			"name": "Abilio Augusto Galvao",
			"father": "Antonio Augusto Galvao",
			"mother": "Perpetua Duarte Galvao",
			"obs": ""
		},
		{
			"folder_id": "581",
			"date": "1908-05-20",
			"name": "Abilio Augusto Magalhaes",
			"father": "",
			"mother": "Maria Jesus Magalhaes",
			"obs": ""
		},
		{
			"folder_id": "581",
			"date": "1908-05-20",
			"name": "Abilio Augusto Magalhaes",
			"father": "",
			"mother": "Maria Jesus Magalhaes",
			"obs": ""
		},
		{
			"folder_id": "581",
			"date": "1908-05-20",
			"name": "Abilio Augusto Magalhaes",
			"father": "",
			"mother": "Maria Jesus Magalhaes",
			"obs": ""
		},
		{
			"folder_id": "569",
			"date": "1869-12-02",
			"name": "Abilio Augusto Santos",
			"father": "Jose Joaquim Santos",
			"mother": "Teresa Jesus",
			"obs": "Antonio Jose Adao,Tio Materno. Proc.12530.   Albino Antonio Ribeiro,Primo Paterno. Proc.12721."
		}
		]
	}    
\end{lstlisting}


\section{Problema 2}

\subsection{index.html}

\includegraphics[width=18cm]{Imagens/"index"}

\subsection{Datas extremas dos registos no dataset}

\includegraphics[width=18cm]{Imagens/"a)"}

\subsection{Distribuição por modalidade em cada ano e no total}

\includegraphics[width=18cm]{Imagens/"b) 1"}
\includegraphics[width=18cm]{Imagens/"b) 2"}

\subsection{Distribuição por ano e género}

\includegraphics[width=18cm]{Imagens/"c)"}

\subsection{Distribuição por morada}

\includegraphics[width=18cm]{Imagens/"d)"}

\subsection{Porcentagem de aptos e não aptos por ano}

\includegraphics[width=18cm]{Imagens/"e)"}



\chapter{Conclusão} \label{concl}
Com a realização deste trabalho deparamo-nos com vários problemas de programação e raciocínio encontrados anteriormente e que sem o recurso/conhecimento das expressões regulares levavam soluções muito mais complexas, o que se traduziu portanto numa valiosa adição à nossa bagagem.
Numa perspetiva geral julgamos estar a entregar um trabalho que cumpria a maioria das nossa metas para o mesmo, no entanto é de salientar que certos aspetos do exercício 2 ficaram à quem das nossas expectativas, nuances tais que serão melhoradas e corrigidas em trabalhos futuros.

\newpage

\section{Bibliografia}
\href{https://pandas.pydata.org/docs/}{Pandas Documentation}
\\
\href{https://matplotlib.org/stable/index.html}{Matplotlib Documentation}


\appendix % apendice

\end{document} 